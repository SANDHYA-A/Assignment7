\documentclass[journal,12pt,twocolumn]{IEEEtran}
\IEEEoverridecommandlockouts
\usepackage{cite}
\usepackage{amsmath,amssymb,amsfonts,bm}
\usepackage{mathtools}
\usepackage{tkz-euclide} 
\usepackage{tikz}
\usetikzlibrary{calc,math}
 \usepackage{caption}
\usepackage{listings}
\usepackage{gensymb}
\let\vec\mathbf
\numberwithin{equation}{subsection}

\newcommand{\myvec}[1]{\ensuremath{\begin{pmatrix}#1\end{pmatrix}}}
\newcommand{\norm}[1]{\left\lVert#1\right\rVert}
\newcommand{\mydet}[1]{\ensuremath{\begin{vmatrix}#1\end{vmatrix}}}

\renewcommand\thesection{\arabic{section}}
\renewcommand\thesubsection{\thesection.\arabic{subsection}}
\renewcommand\thesubsubsection{\thesubsection.\arabic{subsubsection}}

\renewcommand\thesectiondis{\arabic{section}}
\renewcommand\thesubsectiondis{\thesectiondis.\arabic{subsection}}
\renewcommand\thesubsubsectiondis{\thesubsectiondis.\arabic{subsubsection}}
%\renewcommand{\theequation}{\theenumi}
%\numberwithin{equation}{enumi}

\providecommand{\mbf}{\mathbf}
\providecommand{\pr}[1]{\ensuremath{\Pr\left(#1\right)}}
\providecommand{\qfunc}[1]{\ensuremath{Q\left(#1\right)}}
\providecommand{\sbrak}[1]{\ensuremath{{}\left[#1\right]}}
\providecommand{\lsbrak}[1]{\ensuremath{{}\left[#1\right.}}
\providecommand{\rsbrak}[1]{\ensuremath{{}\left.#1\right]}}
\providecommand{\brak}[1]{\ensuremath{\left(#1\right)}}
\providecommand{\lbrak}[1]{\ensuremath{\left(#1\right.}}
\providecommand{\rbrak}[1]{\ensuremath{\left.#1\right)}}
\providecommand{\cbrak}[1]{\ensuremath{\left\{#1\right\}}}
\providecommand{\lcbrak}[1]{\ensuremath{\left\{#1\right.}}
\providecommand{\rcbrak}[1]{\ensuremath{\left.#1\right\}}}

\lstset{
frame=single, 
breaklines=true,
columns=fullflexible
}

\begin{document}

\title{Matrix Theory EE5609 - Assignment 7\\
}

\author{\IEEEauthorblockN{Sandhya Addetla}\\
\IEEEauthorblockA{PhD Artificial Inteligence Department} \\
AI20RESCH14001\\
 }

\maketitle
\begin{abstract}
Find foot of the perpendicular using SVD
\end{abstract}
Download  python code from 
\begin{lstlisting}
https://github.com/SANDHYA-A/Assignment7
\end{lstlisting}
\section{Problem}
Find the foot of the perpendicular for a point on the line of intersection of planes $9x^2 -4y^2 +z^2 -6xz -4y -1 =0$ on to the plane containing the point $(-1, -4 , 3)$.

\section{Solution}
Given the equation of two intersecting planes is 
\begin{align}
   9x^2 -4y^2 +z^2 -6xz -4y -1 =0\label{2.1}
\end{align}
Let the two normals for these planes be $\vec{n_1}$ and $\vec{n_2}$
\begin{align}
   a_{1}x+ b_{1}y +c_{1}z +d_{1} =0\label{2.2}\\
  a_{2}x +b_{2}y +c_{2}z +d_{2} =0\label{2.3}\\
\text{and, }
\vec{n_1}.\vec{n_2} = 0 \label{2.4}
\end{align}
We have,
\begin{align}
   (  a_{1}x+ b_{1}y +c_{1}z +d_{1})( a_{2}x +b_{2}y +c_{2}z +d_{2} ) = 0 \label{2.5}
\end{align}
From equation \ref{2.1} and \ref{2.5} we get the equations of two planes as:
\begin{align}
   ( 3x- 2y -z -1)( 3x +2y-z +1 ) = 0 \label{2.6}\\
\text{The vectors } \vec{n_{1}} \text{ and  }\vec{n_{2}} \text{ are }\notag \\ 
\vec{n_{1}} =\myvec{3 & 2& -1} \label{2.7}\\
\vec{n_{2}} =\myvec{3 & -2& -1} \label{2.8}
\end{align}
From equation \ref{2.6}, we obtain the equation of line of intersection as
\begin{align}
   3x = z \text{ and } y =\frac{-1}{2}\label{2.9}
\end{align}
$\therefore$ The normal perpendicular to the intersection of two planes will be
\begin{align}
   \vec{n} = \vec{n_{1}} \times \vec{n_{2}} \label{2.10}
\end{align}
Substituting eq \ref{2.7} and \ref{2.8} in \ref{2.10} we obtain the normal vector to the intersection of the planes as
\begin{align}
    \vec{n} = \myvec{-4 & 0 & -12} \label{2.11}
\end{align}
Let, $\vec{r}$ be \myvec{x& y& z}. Now, the equation of the plane passing through the point Q(-1, -4 ,3) can be obtained by
\begin{align}
    \vec{n} . ( \vec{r} - \vec{Q}) = 0  \label{2.12}\\
    \myvec{-4 & 0 & -12} \myvec{x-1 \\ y-4 \\ z-3} = 0
\end{align}
Equation of the plane containing the point$(-1, -4 , 3)$ is 
\begin{align}
    \myvec{1 & 0 & 3} \vec{x} = -8 \label{2.14}
\end{align}
Consider a point on the line of intersection at eq \ref{2.9} as $\vec{b} = \myvec{1 & \frac{-1}{2} & 3}$. To find the foot of the perpendicular on to the plane at eq \ref{2.14}, we can use SVD.
We have the two orthogonal vectors $\vec{n_{1}}$ and  $\vec{n_{2}}$ and $\vec{M}$ is the matrix of these orthogonal vectors

We solve the equation
\begin{align}
\vec{M}\vec{x} &= \vec{b}\label{2.15}\\
\intertext{Substituting values of normal vectors and the point on the plane, in \ref{2.15}, We get,}
\myvec{3&3\\2&-2\\-1&-1}\vec{x} &= \myvec{1\\\frac{-1}{2}\\3}
\end{align}
To solve the above equation, we  perform Singular Value Decomposition on $\vec{M}$ as follows,
\begin{align}
\vec{M}=\vec{U}\vec{S}\vec{V}^T \label{2.17}
\end{align}
Where the columns of $\vec{V}$ are the eigen vectors of $\vec{M}^T\vec{M}$ ,the columns of $\vec{U}$ are the eigen vectors of $\vec{M}\vec{M}^T$ and $\vec{S}$ is diagonal matrix of singular value of eigenvalues of $\vec{M}^T\vec{M}$.
\begin{align}
\vec{M}^T\vec{M}=\myvec{14&6\\6&14}\\
\vec{M}\vec{M}^T=\myvec{18&0&-6\\0&8&0\\-6&0&2}
\end{align}
Substituting eq \ref{2.17} in \ref{2.15}, we get,
\begin{align}
\vec{U}\vec{S}\vec{V}^T\vec{x} & = \vec{b}\\
\implies\vec{x} &= \vec{V}\vec{S_+}\vec{U^T}\vec{b} \label{2.21}
\end{align}
Where $\vec{S_+}$ is Moore-Penrose Pseudo-Inverse of $\vec{S}$.
The eigen values of $\vec{M}\vec{M}^T$ are obtained as below,
\begin{align}
\mydet{\vec{M}\vec{M}^T - \lambda\vec{I}} &= 0\\
\implies\begin{vmatrix}18-\lambda&0&-6\\0&8-\lambda&0\\-6&0&2-\lambda\end{vmatrix} &=0\\
\implies-\lambda^3+28 \lambda^2-160\lambda &=0
\end{align}
The eigen values of $\vec{M}\vec{M}^T$ are,
\begin{align}
\lambda_1 &=20\\
\lambda_2 &= 8\\
\lambda_3 &=0
\end{align}
Eigen vectors of $\vec{M}\vec{M}^T$ are,
\begin{align}
\vec{u}_1=\myvec{-3\\0\\1},
\vec{u}_2=\myvec{0\\1\\0},
\vec{u}_3=\myvec{\frac{1}{3}\\0\\1}
\end{align}
Normalizing the eigen vectors we get,
\begin{align}
\vec{u}_1=\myvec{\frac{-3}{\sqrt{10}}\\0\\\frac{1}{\sqrt{10}}},
\vec{u}_2=\myvec{0\\1\\0},
\vec{u}_3=\myvec{\frac{1}{\sqrt{10}}\\0\\\frac{3}{\sqrt{10}}}
\end{align}
$\vec{U}$ is obtained as  follows,
\begin{align}
\myvec{\frac{-3}{\sqrt{10}}& 0&\frac{1}{\sqrt{10}}\\
0&1&0\\
\frac{1}{\sqrt{10}}&0&\frac{3}{\sqrt{10}}}  \label{2.30}
\end{align}
After computing the singular values from eigen values $\lambda_1, \lambda_2, \lambda_3$ we get $\vec{S}$ of  as follows,
\begin{align}
\vec{S}=\myvec{\sqrt{20}&0\\0&\sqrt{8}\\0&0} \label{2.31}
\end{align}
The eigen values of $\vec{M}^T\vec{M}$ are obtained as below,
\begin{align}
\mydet{\vec{M}^T\vec{M} - \lambda\vec{I}} &= 0\\
\implies\begin{vmatrix}14-\lambda&6\\6&14-\lambda \end{vmatrix}&=0\\
\implies\lambda^2-28\lambda+160 &=0
\end{align}
The eigen values of $\vec{M}^T\vec{M}$ are,
\begin{align}
\lambda_4 &= 20\\
\lambda_5 &=8
\end{align}
Eigen vectors of $\vec{M}^T\vec{M}$ are,
\begin{align}
\vec{v}_1=\myvec{1\\1},
\vec{v}_2=\myvec{-1\\1}
\intertext{Normalizing the eigen vectors we get,}
\vec{v}_1=\myvec{\frac{1}{\sqrt{2}}\\\frac{1}{\sqrt{2}}},
\vec{v}_2=\myvec{-\frac{1}{\sqrt{2}}\\\frac{1}{\sqrt{2}}}
\end{align}
 $\vec{V}$ is obtained as follows,
\begin{align}
\vec{V}=\myvec{\frac{1}{\sqrt{2}}&-\frac{1}{\sqrt{2}}\\\frac{1}{\sqrt{2}}&\frac{1}{\sqrt{2}}} \label{2.39}
\end{align}
From eq \ref{2.30}, \ref {2.31} and \ref{2.39}, $\vec{M}$ as Singular Value Decomposition can be written as,
\begin{align}
\vec{M} =\myvec{\frac{-3}{\sqrt{10}}& 0&\frac{1}{\sqrt{10}}\\0&1&0\\ \frac{1}{\sqrt{10}}&0&\frac{3}{\sqrt{10}}}\myvec{\sqrt{20}&0\\0&\sqrt{8}\\0&0} \myvec{\frac{1}{\sqrt{2}}&-\frac{1}{\sqrt{2}}\\\frac{1}{\sqrt{2}}&\frac{1}{\sqrt{2}}}^T
\end{align}
Moore-Penrose Pseudo inverse of $\vec{S}$ is obtained as,
\begin{align}
\vec{S_+} = \myvec{\frac{1}{\sqrt{20}}&0&0\\0&\frac{1}{\sqrt{8}}&0}  \label{2.41}
\end{align}
Using equation \ref{2.15}, \ref{2.30} and \ref{2.39} in equation \ref{2.21} we obtain value of $\vec{x}$ as:
\begin{align}
\vec{U}^T\vec{b}=\myvec{0 \\ \frac{-1}{2}\\ \sqrt{10}}\\
\vec{S_+}\vec{U}^T\vec{b}=\myvec{0\\ \frac{-1}{4\sqrt{2}}}\\
\vec{x} = \vec{V}\vec{S_+}\vec{U}^T\vec{b} = \myvec{\frac{1}{8}\\\frac{-1}{8}} \label{2.44}
\end{align}
We can verify the obtained solution,
\begin{align}
\vec{M}^T\vec{M}\vec{x} = \vec{M}^T\vec{b} \label{2.45}
\end{align}
Computing the RHS values in equation \ref{2.45}, we get,
\begin{align}
\vec{M}^T\vec{M}\vec{x} &= \myvec{-1\\1}\\
\implies\myvec{14&6\\6&14}\vec{x} &= \myvec{-1\\1} \label{2.47}
\end{align}
Solving the augmented matrix from eq \ref{2.47} we get,
\begin{align}
\myvec{14&6&-1\\6&14&1} &\xleftrightarrow{R_1=\frac{1}{14}R_1}\myvec{1&\frac{3}{7}&\frac{-1}{14}\\6&14&1}\\
&\xleftrightarrow{R_2=R_2-6R_1}\myvec{1&\frac{3}{7}&-\frac{1}{14}\\0&\frac{80}{7}&-\frac{10}{7}}\\
&\xleftrightarrow{R_2=\frac{7}{80}R_2}\myvec{1&\frac{3}{7}&-\frac{1}{14}\\0&1&\frac{1}{8}}\\
&\xleftrightarrow{R_1=R_1-\frac{3}{7}R_2}\myvec{1&0&\frac{-1}{8}\\0&1&\frac{1}{8}} \label{2.51}
\end{align}
The value of $\vec{x}$ is obtained from eq \ref{2.51} as
\begin{align}
\vec{x}=\myvec{\frac{1}{8}\\\frac{-1}{8}} \label{2.52}
\end{align}
From equations \ref{2.44} and \ref{2.52} the solution is verified.
\end{document}
